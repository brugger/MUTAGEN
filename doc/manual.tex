\documentclass[a4paper,10pt, onecolumn]{article}
\setlength{\textwidth}{17cm}
\setlength{\textheight}{24cm}
\setlength{\topmargin}{-1.5cm}
\setlength{\evensidemargin}{-5mm}
\setlength{\oddsidemargin}{-5mm}

\title{MUTAGEN: Multi User Tool for Annotating GENomes\\~\\DOCUMENTATION/MANUAL}

\begin{document}
\maketitle

\center{Revised for MUTAGEN version 4.xx}


\begin{center}
Kim Br�gger$^1$, Peter Redder$^1$ and Marie Skovgaard$^2$\\~\\~\\
\end{center}

1:
Danish Archaea Centre,
Institute of Molecular Biology,
University of Copenhagen,
S�lvgade 83H,
DK-1307 Copenhagen K,
Denmark

2:
Center for Biological Sequence Analysis,
The Technical University of Denmark,
Building 208,
DK-2800 Lyngby,
Denmark



\newpage
\tableofcontents
\newpage

\section{Introduction}

\subsection{Word explanations}
\label{word}

This list explain some of the commonly used phrases in this manual

\begin{itemize}

\item Administrator: The administrator(s) is responsible for the
database behind the web-interface. The administrator(s) sets up the
database and makes updates.

\item Annotator: A person who is allowed to write annotations in the
system. Several annotators can make annotations for the same gene.

\item Contig: An assemblied sequence fragment. At the end of the
sequencing project one contig will cover the full chromosome.

\item Gene finder: A software that identifies putative genes in the
sequence. This can either be identification of ORFs or a more advanced
tool such as EasyGene. If an ORF finder is used, a useful score can be
generated by dividing 250 with the ORF length (the lower the score the
more probable the gene), and set the cutoff at '5'.

\item Homology browser: A sequence browser that reveals the homology
between regions in related organisms. The homology browser can be used
to locate conserved operons or see if questionable genes are found in
the same contexts in other organisms.

\item Master annotator: An annotator who is allowed to finalise an
annotation and thereby close the gene.

\item ORF browser: A visualisation tool showing a part of the sequence
with the expected genes colour coded for database matches. The ORF
browser gives an overview of the genes in an area and can be used as a
starting point for annotation.

\item Sequence browser: A six-frame translation of a fragment of the
sequence, this allows the annotator to identify proper start codons
and possible frameshifts.

\end{itemize}

\subsection{Requirements}
\label{requirements}

The system relies on several programs and libraries that need to be
installed in order to get the annotation system up and running.

\subsubsection{Programs}
Please follow the installation guidelines outlined for the different
programs.

\begin{itemize}
\item A http server, we have been using apache on all setups \\
	(\verb|http://httpd.apache.org/|).
\item A MySQL database server for information storage, the system have
      been tested with the version 3.23.-\\
	(\verb|http://www.mysql.com/|).
\item Perl, the main programming language used in MUTAGEN\\
	(\verb|http://www.cpan.org/|)
\item NCBI BLAST for searching through sequences \\
	(\verb|ftp://ftp.ncbi.nih.gov/blast/executables/|)
\item T\_coffee for making multiple alignments with homologous proteins \\
  	(\verb|http://igs-server.cnrs-mrs.fr/~cnotred/Projects_home_page/t_coffee_home_page.html|)
\item overLIB, used for displaying information on navigations maps \\
	(\verb|http://www.bosrup.com/web/overlib/|).
\item tRNAscan-SE locates tRNA's \\
	(\verb|http://www.genetics.wustl.edu/eddy/tRNAscan-SE/|)
\item HMMR for profile HMMs \\
	(\verb|http://hmmer.wustl.edu/|)
\item perl LWP module.
\end{itemize}

    
\subsubsection{Libraries}

The following perl libraries are used, some of these come with the
standard perl installation package, others should be downloaded and
installed separately.

\begin{itemize}
\item DBI (General Database interface )
\item DBD-mysql (Interface to the MySQL database).
\item GD (graphic library linked to libgd)
\item CGI (Common Gateway Interface for the web)
\item Data (Used for debugging)
\item Digest (Calculates the unique session ids)
\item HTTP (Used for running things on the web)
\item LWP (Used for running things on the web)
\item Net (Used for identifying where the user connects from)
\item Socket (Used for identifying where the user connects from)
\item POSIX (Common functions)
\end{itemize}

\subsubsection{Databases}

The system relies on a selection of databases that can be installed
according to personal preferences. Several can be downloaded from the
MUTAGEN homepage, where we have formatted the databases to the format
used in MUTAGEN. The rest of the databases have to be downloaded from
their own sites.

\begin{itemize}
\item SWISS-PROT \verb|http://dac.molbio.ku.dk/bioinformatics/MUTAGEN/download.html|
\item COG \verb|http://dac.molbio.ku.dk/bioinformatics/MUTAGEN/download.html|
\item NCBI genomes  \verb|http://dac.molbio.ku.dk/bioinformatics/MUTAGEN/download.html|
\item nr NCBI (\verb|ftp://ftp.ncbi.nih.gov/blast/db/FormattedDatabases/nr.tar.gz|)
\item Pfam\_ls (\verb|ftp://ftp.genetics.wustl.edu/pub/Pfam/Pfam_ls.gz|)
\end{itemize}

There exists 3 different fractions of the NCBI genomes. One contain
only archaeal genomes, the second only bacterial genomes and the last
all prokaryotic genomes. You should only download and install one of
these packages, otherwise important information is deleted. All the
databases should be installed in the folder named \verb|blastdb|.

\section{Installation instructions}
\label{instal}
\subsection{How to install MUTAGEN}

When the programs and liberies mentioned above is downloaded and the
different programs and paths have been setup, it is possible to
install MUTAGEN. First unpack MUTAGEN:

\begin{verbatim}
> tar zxvf MUTAGEN_xxxx.tgz
\end{verbatim}

this will create a library called MUTAGEN_xx where the scripts and
modules are placed. Go into that library and follow the instructions
described below. But first you have to create a database and allow
access for a user.

\subsection{Creation of the database.}

A mysql database has to be correctly installed and configured on the
system, before this step can be achieved. Here we are going to create
a database (dbname) with a user login (dbuser) using the password
(dbpass):

\begin{verbatim}
> mysql --user=root mysql -p
Enter password: 
Welcome to the MySQL monitor.  Commands end with ; or \g.
Your MySQL connection id is 3 to server version: 3.23.34a

Type 'help;' or '\h' for help. Type '\c' to clear the buffer

mysql> CREATE DATABASE dbname;
Query OK, 1 row affected (1.11 sec)

mysql> GRANT ALL PRIVILEGES ON dbname.* TO dbuser@localhost IDENTIFIED
     > BY 'dbpass' WITH GRANT OPTION;
Query OK, 1 row affected (1.01 sec)
\end{verbatim}

A new user \verb|dbname| with password \verb|dbname| has now been
created. The user only has access to the database named dbname from
the localhost. Try to login with

\begin{verbatim}
> mysql -udbuser  -pdbpass dbname
\end{verbatim}

If this does not work something have been done incorrectly, if so
retrace your steps and get it right this time. To create the database
use the following command:

\begin{verbatim}
> mysql -udbuser  -pdbpass dbname < sql/tables.sql
\end{verbatim}

Some default values are needed for instalation of the database, these
are added with

\begin{verbatim}
> mysql -u dbuser  -pdbpass dbname < sql/values.sql
\end{verbatim}

The database is now up and running.

\subsection{Installing external programs}
\subsection{Setting up the perl scripts}

MUTAGEN needs to know the path to where the system was installed and
other important stuff, these values can easily be given using the
install.pl script placed in the root of the installation. This script
will setup several things, including a path to the modules and
information regarding the database. After that you should check the
configuration file \verb|modules/conf.pm| where you should check the
location of locally installed programmes. Furthermore you can also
alter some of the behaviour of the program here, such as if guest
should be allow and where the program should store its log.

\subsection{Handling data and the database}

Unlike previous version of MUTAGEN it is now possible to control the
behaviour and integration of data from the web pages. When the system
is installed there is created a single user: \verb|admin|. To login on
the system as use \verb|admin| as login and password. You are now
administrator, and can control the system. At the top there is a menu
bar, and when logged in as a member of the administration group an
extra menu gets activated: \verb|Admin|. Click this link and you will
get access to all the administrative functions. First thing I suggest
you to do is create a new user and add him/her to the admin group and
delete the orginal admin. This is to ensure that no one else logs in
as administrator and alters things.


\subsection{Uploading sequences in the database.}

By selecting \verb|Handle sequences| sub menu the possibility to
handle the sequences are revealed. Select the \verb|Upload sequence|
button, and you will see a new menu. Here you are presented with a few
fields to fill out and a single selection. You can either select to
upload a file just containing DNA in the form of a fasta formated
file. Or upload a file in the GenBank format where annotations and
genes are transfered to the system. If you upload a GenBank file I
greatly recommend that you look at the code in
\verb|modules/page/admin/sequence.pm| where it is decided what
information is stored where. The problem with GenBank formatted files
are that not 2 files are identically formatted, but if you want to
upload GenBank files just be aware of this.

After you press the submit button, a organism entry, version control
and the sequence is uploaded and stored in the database. 

\subsection{Editing a sequence in the database.}

One of the new feature in this version is that it is now possible to
edit a sequence in the database. This allows for altering a sequence
without having to identify genes again and transfer information
between 2 different versions of a sequence. Generally keeping things
simple makes life a bit easier, even though they way to this might be
difficult.

\subsection{Delete a sequence in the database.}

Here you can delete a sequence and all the information relate to this
sequence.

\subsection{Upload sequence info}

Here is the possibility for uploading information about genes for a
sequence. I generally recommend that you transfer the gene information
using the GFF2 format. There is a document describing this format in
the \verb|docs| directory. Currently it is only possible to upload
genes for a single sequence at once, but uploading a set of genes for
a organism will come eventually. This will not currently make it
possible to see the genes, this is because the system can handle
several genefinders at once as described later in this document.  I
find it very important to keep the name of the genefinder source
strict since this will be useful later, but if you only use a single
genefinder then this is not important. The MUTAGEN distribution
contains a genefinder in the tools sub directory called \verbn|ORFinder.pl|.


\subsection{Finding genes}

Select the \verb|(Re-)Find genes| first select a sequence that you
want to find genes belonging to. Here it becomes possible to rate the
different genefinders used on this sequence. I will try do describe
this in greater detail later, otherwise if in doubt send me a mail and
I will try to explain it better.

\subsection{Build local databases}

To be able to blast against the data located in the internal databases
you have to export these sequences. This is done with the
\verb|Build local DBases| button. I do not think there is much more to
say about this at this point in time, perhaps some more later.

\subsection{Gene clusters}

Before you can use the gene clustering information you have to
calculate the cluster information, and extract clusters from this
information and store it in the database. The calculation of the
cluster basicly consists of blasting the all the sequences against
each other. Ones this report have been produced (might take a while
based on the amount of data), you can then cluster the genes
together. Here you have several options for how strict the clustering
should be: Each gene is compared to the rest of the genes by comparing
them one by one.


\subsection{}

\subsection{}

\subsection{}

\subsection{}

\subsection{}

\subsection{}

\subsection{}

\subsection{}



\subsection{First login on the database}

Login on the system using \verb|admin| as login and no password. You
are now administrator, and can create and delete users and link users
to groups. Create a new user and link him/her to the admin group and
delete the orginal admin. The system should now be safe and not wide
open for everyone to use.

%\newpage
%\section{Advanced installation}
%\subsection{Installation of external programs}
%\subsection{Installation of perl modules}
%\subsection{Installation of databases}

%\newpage
%\section{System administration}
%\subsection{Uploading sequence}
%\subsection{Running prediction methods}
%\subsection{Etc...}
%\subsection{Adding users to groups for increased functionality etc}
%\subsubsection{security, and how it (should) work}

\section{Administrator pages}
\label{admin}
\subsection{Users Page}

This is where the system administrator(s) handle users and their
privileges. It is possible to: Create, edit or delete users. Create or
delete groups with different privileges. Link or unlink the users and
the different groups which gives the user access to the different
functionalities in the system. A user should often be member of
several groups to gain access to all functionalities.

\textbf{Guest}: Only access to BLAST (\ref{blast}) and search (\ref{search})
\textbf{Annotation:}: Access to the annotation tool as well as guest access. \textbf{Master:} Is allowed to finalise annotions.
\textbf{Admin:} Administrator privileges (\ref{admin}).  
\textbf{Genbank:} Can annotate in sequences downloaded from GenBank (\ref{gbk}).
\textbf{Comparison:} Access to the sequence comparison tool (\ref{comparison}).

\subsection{System Administration Page}
The majority of the data in the system is handled from this page. By a click
on a button, it is possible to: 
\begin{itemize}
%\item Recalculate homology dependencies 
%\item Add new unannotated sequence
%\item Add a new genbank file
\item Build BLAST databases 
%\item Select genomes to show on the BLAST page
\item Run predictions for sequences in the database
%\item Update external data
\item Remove organism.
\end{itemize}

\section{Using the system.}
\label{use}
\subsection{Main page}
The main page consists of a menu-bar and corresponding short explanation of each item on it. The Menu-bar is found on many of the other pages and it contains links to the various tools available in MUTAGEN.

\subsection{Annotation Tool Page}
The annotation tool page contains a list of the projects in the
system. A specific project is selected and on the following page the
user is asked to choose a contig from a list of contigs (the list will
only contain one contig, if the sequence is finished). The user can
also choose a start and stop position, this is useful for long contigs
where a distance of about 20-40 kbp is usually good for most monitor
resolutions. Furthermore, a maximum gene finder score is needed. This
value should be set to 5 if the most likely genes should be shown
(\ref{word}). The maximum of this score depends on the chosen
treshhold. Showing hidden genes will include 'hidden genes' on the
graphical presentation. Submit will open the Annotation Main Page
(\ref{AMP}).

\subsubsection{Annotation Main Page}
\label{AMP}
This page shows the sequence browser at the top, and information about
a chosen gene at the bottom of the page. The colour code of the genes
are revealed by clicking on the Colour code link. If the chosen region
is too large or too small a new image can be generated using new
positions by clicking on the Zoom link. It can sometimes be desirable
to add a DNA feature (transposons, pathogenisity islands, RNA genes
etc.) or an ORF. This can be done by clicking on the Add ORF/feature
link. The blue arrows at each end of the sequence will move the image
to the next sequence of the same length. By clicking on an ORF (the
bars in the picture) the information about the ORF will be revealed as
well as a links to the Annotation Page, amino acid and nucleotide
sequences, and flanking sequence of the ORF. From this page it is also
possible to hide an ORF, so that it will only appear if asked for,
this is chosen instead of deletion to avoid deletion of valuble
information.

%\item Hide genes


A summary of the collected data and existing manual annotations will
appear on the lower half of the screen. It is also possible to see the
output from prediction method and the best hit from alignment to
different databases. The full database searches can be seen by
clicking on the results in the table. If the entry 'homology' is
present there are similar genes present in this or other projects in
the database, and the link opens the homology browser (\ref{homobrow})
where they can be viewed.

%SCREEN SHOT of Annotation main page with an ORF info table

\subsubsection{Annotation Page}
\label{ann_page}
This is the page where the manual annotations are written. The page
consists of a number of buttons at the top that opens different BLAST
reports and other prediction tool reports. A number of fields that can
be filled out by the annotator I shown on the page. A summary of the
collected data is found to the right of the page.

\subsubsection{Homology browser}
\label{homobrow}
The homology browser can be chosen if homologs exist in the
database. This tools allows for regional comparison between related
species. By clicking on \verb|x copies in database| link information
of the copies will appear and either a multiple alignement or the
homology browser can be viewed.

\subsubsection{Help page}
\label{help}
The help page is meant as a guide to what the different fields on the
annotation page means. The page can easily be extended to contain
information of the collected data, such as useful thresholds or how
the data was generated.

%\begin{itemize}
%\item Sequence alignment
%\end{itemize}


\subsection{BLAST}
\label{blast}
This page allows the user to BLAST searches of nucleotide or amino
acid sequences against combinations of the sequences in the database.

\subsection{GenBank page}
\label{gbk}
These pages have functionalities that are similar to those of the the
annotation page. The data is extracted from finished genomes
downloaded from NCBI. It is possible to analyse the sequences in a
similar manner as the ongoing sequencing projects in the database.

\subsection{Sequence Export Page}
\label{seqexport} 
The sequence export page allows the user to export nucleotide
sequences and/or all the genes from a given project, either in fasta
or GenBank format. It is also possible to get a section of a sequence
by typing start and stop nucleotide.

\subsection{Project members}
Here, each member of the project can be listed with login, e-mail, and
homepage if available. The entries are taken from the User Profile
Page \ref{UPP}. E-mail addresses and homepages are shown as links.

\subsection{User Profile Page}
\label{UPP}
Each user can change their personal setup. Name, e-mail and homepage
can be entered in the user profile page and will be shown on the
Project Members page. Login however, cannot be changed.

%It is on this page also possible to change the default values of  several %constants in MUTAGEN to the users own personal preferences such as background %colour (with a link to a colour scheme), default gene finder score, default %minimum gene length, and e-value for the BLAST reports in the collected data.

\subsection{Annotation status}
The annotation progress for the projects in the database is shown,
giving the total number of genes, genes annotated and genes
finalised. User statistics include number of annotations by each user
in each project and the date of the last login.

%The 'Graphical presentation' link will generate an up-to-date histogram of the  %annotation progress in total, and the number of genes annotated each day.

\subsection{Sequence comparison Page}
\label{comparison}
This is where whole sequences can be compared to each other, this
program is only useful when having sequences shorter than 70 KB,
otherwise use the genomic neighbourhood as described above. The
desired sequences, 2 or more, can be clicked off. The minimum length
of the genes shown can also be set, to avoid too many genes in the
picture. Submit opens the homology browser (\ref{homobrow}) where each
gene and its homologues are indicated with the same colour and
pattern.

\subsection{Search}
\label{search}
This page gives the option of searching in the annotations: Type in a
word, a part of a word, a combination of words or a gene-id (as
\verb|gid:xxx|). Choose if the search should be done in manual
annotations and/or the automaticly collected data and press
submit. The search results now shows how many of the words were found
in each entry. There are 2 links for each hit, the first link gives
the ORF information, the other link goes to the ORF browser, where the
ORF in question is highlighted in yellow.

\subsection{Bug report}
\label{bug}
Bugs can be reported to the system administrator(s). Radio-buttons
allows the user to select which admin to report to. Default is all of
them.

\subsection{News}
\label{news}
This is a link reserved for communications from the system
administrator(s) to the rest of the project members.

\subsection{Manual}
\label{manual}
A link to this manual.

\subsection{About MUTAGEN}
This link contains information about the software and copyright
information.

%\newpage
%\section{Development}
%\subsection{Structure of the MUTAGEN modules}
%\subsection{Data format of input files}
%\subsection{Adding new fields to the automatic data collection}

\newpage
\appendix

\section*{Appendix}

\section{Copyright notice}

		    GNU GENERAL PUBLIC LICENSE
   TERMS AND CONDITIONS FOR COPYING, DISTRIBUTION AND MODIFICATION

  0. This License applies to any program or other work which contains
a notice placed by the copyright holder saying it may be distributed
under the terms of this General Public License.  The "Program", below,
refers to any such program or work, and a "work based on the Program"
means either the Program or any derivative work under copyright law:
that is to say, a work containing the Program or a portion of it,
either verbatim or with modifications and/or translated into another
language.  (Hereinafter, translation is included without limitation in
the term "modification".)  Each licensee is addressed as "you".

Activities other than copying, distribution and modification are not
covered by this License; they are outside its scope.  The act of
running the Program is not restricted, and the output from the Program
is covered only if its contents constitute a work based on the
Program (independent of having been made by running the Program).
Whether that is true depends on what the Program does.

  1. You may copy and distribute verbatim copies of the Program's
source code as you receive it, in any medium, provided that you
conspicuously and appropriately publish on each copy an appropriate
copyright notice and disclaimer of warranty; keep intact all the
notices that refer to this License and to the absence of any warranty;
and give any other recipients of the Program a copy of this License
along with the Program.

You may charge a fee for the physical act of transferring a copy, and
you may at your option offer warranty protection in exchange for a fee.

  2. You may modify your copy or copies of the Program or any portion
of it, thus forming a work based on the Program, and copy and
distribute such modifications or work under the terms of Section 1
above, provided that you also meet all of these conditions:

    a) You must cause the modified files to carry prominent notices
    stating that you changed the files and the date of any change.

    b) You must cause any work that you distribute or publish, that in
    whole or in part contains or is derived from the Program or any
    part thereof, to be licensed as a whole at no charge to all third
    parties under the terms of this License.

    c) If the modified program normally reads commands interactively
    when run, you must cause it, when started running for such
    interactive use in the most ordinary way, to print or display an
    announcement including an appropriate copyright notice and a
    notice that there is no warranty (or else, saying that you provide
    a warranty) and that users may redistribute the program under
    these conditions, and telling the user how to view a copy of this
    License.  (Exception: if the Program itself is interactive but
    does not normally print such an announcement, your work based on
    the Program is not required to print an announcement.)

These requirements apply to the modified work as a whole.  If
identifiable sections of that work are not derived from the Program,
and can be reasonably considered independent and separate works in
themselves, then this License, and its terms, do not apply to those
sections when you distribute them as separate works.  But when you
distribute the same sections as part of a whole which is a work based
on the Program, the distribution of the whole must be on the terms of
this License, whose permissions for other licensees extend to the
entire whole, and thus to each and every part regardless of who wrote it.

Thus, it is not the intent of this section to claim rights or contest
your rights to work written entirely by you; rather, the intent is to
exercise the right to control the distribution of derivative or
collective works based on the Program.

In addition, mere aggregation of another work not based on the Program
with the Program (or with a work based on the Program) on a volume of
a storage or distribution medium does not bring the other work under
the scope of this License.

  3. You may copy and distribute the Program (or a work based on it,
under Section 2) in object code or executable form under the terms of
Sections 1 and 2 above provided that you also do one of the following:

    a) Accompany it with the complete corresponding machine-readable
    source code, which must be distributed under the terms of Sections
    1 and 2 above on a medium customarily used for software interchange; or,

    b) Accompany it with a written offer, valid for at least three
    years, to give any third party, for a charge no more than your
    cost of physically performing source distribution, a complete
    machine-readable copy of the corresponding source code, to be
    distributed under the terms of Sections 1 and 2 above on a medium
    customarily used for software interchange; or,

    c) Accompany it with the information you received as to the offer
    to distribute corresponding source code.  (This alternative is
    allowed only for noncommercial distribution and only if you
    received the program in object code or executable form with such
    an offer, in accord with Subsection b above.)

The source code for a work means the preferred form of the work for
making modifications to it.  For an executable work, complete source
code means all the source code for all modules it contains, plus any
associated interface definition files, plus the scripts used to
control compilation and installation of the executable.  However, as a
special exception, the source code distributed need not include
anything that is normally distributed (in either source or binary
form) with the major components (compiler, kernel, and so on) of the
operating system on which the executable runs, unless that component
itself accompanies the executable.

If distribution of executable or object code is made by offering
access to copy from a designated place, then offering equivalent
access to copy the source code from the same place counts as
distribution of the source code, even though third parties are not
compelled to copy the source along with the object code.

  4. You may not copy, modify, sublicense, or distribute the Program
except as expressly provided under this License.  Any attempt
otherwise to copy, modify, sublicense or distribute the Program is
void, and will automatically terminate your rights under this License.
However, parties who have received copies, or rights, from you under
this License will not have their licenses terminated so long as such
parties remain in full compliance.

  5. You are not required to accept this License, since you have not
signed it.  However, nothing else grants you permission to modify or
distribute the Program or its derivative works.  These actions are
prohibited by law if you do not accept this License.  Therefore, by
modifying or distributing the Program (or any work based on the
Program), you indicate your acceptance of this License to do so, and
all its terms and conditions for copying, distributing or modifying
the Program or works based on it.

  6. Each time you redistribute the Program (or any work based on the
Program), the recipient automatically receives a license from the
original licensor to copy, distribute or modify the Program subject to
these terms and conditions.  You may not impose any further
restrictions on the recipients' exercise of the rights granted herein.
You are not responsible for enforcing compliance by third parties to
this License.

  7. If, as a consequence of a court judgment or allegation of patent
infringement or for any other reason (not limited to patent issues),
conditions are imposed on you (whether by court order, agreement or
otherwise) that contradict the conditions of this License, they do not
excuse you from the conditions of this License.  If you cannot
distribute so as to satisfy simultaneously your obligations under this
License and any other pertinent obligations, then as a consequence you
may not distribute the Program at all.  For example, if a patent
license would not permit royalty-free redistribution of the Program by
all those who receive copies directly or indirectly through you, then
the only way you could satisfy both it and this License would be to
refrain entirely from distribution of the Program.

If any portion of this section is held invalid or unenforceable under
any particular circumstance, the balance of the section is intended to
apply and the section as a whole is intended to apply in other
circumstances.

It is not the purpose of this section to induce you to infringe any
patents or other property right claims or to contest validity of any
such claims; this section has the sole purpose of protecting the
integrity of the free software distribution system, which is
implemented by public license practices.  Many people have made
generous contributions to the wide range of software distributed
through that system in reliance on consistent application of that
system; it is up to the author/donor to decide if he or she is willing
to distribute software through any other system and a licensee cannot
impose that choice.

This section is intended to make thoroughly clear what is believed to
be a consequence of the rest of this License.

  8. If the distribution and/or use of the Program is restricted in
certain countries either by patents or by copyrighted interfaces, the
original copyright holder who places the Program under this License
may add an explicit geographical distribution limitation excluding
those countries, so that distribution is permitted only in or among
countries not thus excluded.  In such case, this License incorporates
the limitation as if written in the body of this License.

  9. The Free Software Foundation may publish revised and/or new versions
of the General Public License from time to time.  Such new versions will
be similar in spirit to the present version, but may differ in detail to
address new problems or concerns.

Each version is given a distinguishing version number.  If the Program
specifies a version number of this License which applies to it and "any
later version", you have the option of following the terms and conditions
either of that version or of any later version published by the Free
Software Foundation.  If the Program does not specify a version number of
this License, you may choose any version ever published by the Free Software
Foundation.

  10. If you wish to incorporate parts of the Program into other free
programs whose distribution conditions are different, write to the author
to ask for permission.  For software which is copyrighted by the Free
Software Foundation, write to the Free Software Foundation; we sometimes
make exceptions for this.  Our decision will be guided by the two goals
of preserving the free status of all derivatives of our free software and
of promoting the sharing and reuse of software generally.

			    NO WARRANTY

  11. BECAUSE THE PROGRAM IS LICENSED FREE OF CHARGE, THERE IS NO WARRANTY
FOR THE PROGRAM, TO THE EXTENT PERMITTED BY APPLICABLE LAW.  EXCEPT WHEN
OTHERWISE STATED IN WRITING THE COPYRIGHT HOLDERS AND/OR OTHER PARTIES
PROVIDE THE PROGRAM "AS IS" WITHOUT WARRANTY OF ANY KIND, EITHER EXPRESSED
OR IMPLIED, INCLUDING, BUT NOT LIMITED TO, THE IMPLIED WARRANTIES OF
MERCHANTABILITY AND FITNESS FOR A PARTICULAR PURPOSE.  THE ENTIRE RISK AS
TO THE QUALITY AND PERFORMANCE OF THE PROGRAM IS WITH YOU.  SHOULD THE
PROGRAM PROVE DEFECTIVE, YOU ASSUME THE COST OF ALL NECESSARY SERVICING,
REPAIR OR CORRECTION.

  12. IN NO EVENT UNLESS REQUIRED BY APPLICABLE LAW OR AGREED TO IN WRITING
WILL ANY COPYRIGHT HOLDER, OR ANY OTHER PARTY WHO MAY MODIFY AND/OR
REDISTRIBUTE THE PROGRAM AS PERMITTED ABOVE, BE LIABLE TO YOU FOR DAMAGES,
INCLUDING ANY GENERAL, SPECIAL, INCIDENTAL OR CONSEQUENTIAL DAMAGES ARISING
OUT OF THE USE OR INABILITY TO USE THE PROGRAM (INCLUDING BUT NOT LIMITED
TO LOSS OF DATA OR DATA BEING RENDERED INACCURATE OR LOSSES SUSTAINED BY
YOU OR THIRD PARTIES OR A FAILURE OF THE PROGRAM TO OPERATE WITH ANY OTHER
PROGRAMS), EVEN IF SUCH HOLDER OR OTHER PARTY HAS BEEN ADVISED OF THE
POSSIBILITY OF SUCH DAMAGES.

		     END OF TERMS AND CONDITIONS


\end{document}


